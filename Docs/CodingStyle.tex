%------------------------------------------------------------------------------
% File Name : test.tex
%
% Purpose : Define coding standards for Pandora3D.
%
% Creation Date : 2010-04-14
%
% Last Modified : on. 14. april 2010 kl. 14.52 +0200
%
% Created By :  Martin Erts�s
%------------------------------------------------------------------------------
\documentclass[a4paper,12pt]{article}
\usepackage[T1]{fontenc}
\usepackage[latin1]{inputenc}
\usepackage{enumerate}

\title{Coding style for Pandora3D}
\author{Martin Erts�s\\ martiert@ifi.uio.no}

\begin{document}
\maketitle

\section{File endings}

\begin{enumerate}[(i)]
\item All header files should have the file ending .h.
\item If the implementation of a header file is to be in the header
file, this implementation should be in a .inl file which is included in
the header file.
\item All files which should be compiled, is to be in a .cpp file.
\end{enumerate}

\section{Header files}

\begin{enumerate}[(i)]
\item All header files have to define the name PANDORA<FILENAME> in
capital letters.
\item All header files have to use the Pandora namespace, possibly along
with other namespaces that tells what kind of purpose it has. Like all
math headers should use the Pandora::Math namespace.
\item All header files have to be documented well using doxygen. All
functions is to be documented, along with all classes and structs. See
the section about doxygen style to see the documentation style used.
\item All header files should start with the header PandoraHeader.txt
from the Headers folder.
\item All typedefs you wish to put in a header, should be written in a
.inl file with the same name as the header file, and included before the
end of the innermost namespace.
\item Headers can \textbf{NOT} use the using keyword.
\end{enumerate}

\section{Inline files}

\begin{enumerate}[(i)]
\item All inline files should use the author.txt header from the Headers
folder, which is to be updated when any changes are done to the file
before commiting the changes.
\item Inline files should be written without use of any Pandora
namespaces.
\item Inline files can \textbf{NOT} use the using keyword.
\item All typedefs is to go at the top of the file, right under the
header.
\end{enumerate}

\section{Cpp files}

\begin{enumerate}[(i)]
\item cpp files should use the author.txt header from the Headers
folder, which is to be updated when any changes are done to the file
before commiting the changes.
\item All includes should be done at the top of the file.
\item One should not use the using keyword for whole namespaces, but
specify which functions/classes/structs to use. Using keywords should
also go at the top of the file.
\end{enumerate}

\section{Documentation}

\begin{enumerate}[(i)]
\item All header files are documented using doxygen.
\item The documentation comments starts with /** and ends with */.
\item One uses \textit{\textbackslash param} for parameters to the function. One
for each parameter. After the name of the parameter there should be a
space, then a -, then another space before the description.
\item One uses \textit{\textbackslash return} for return values from the
function.
\item One uses \textit{\textbackslash note} for notes about the function.
\end{enumerate}

\section{Git}

\begin{enumerate}[(i)]
\item All commits to the master branch should compile and run without
problems.
\item All commits should have a descriptive message containing changes
done in this commit.
\item Everything in the master branch should pass the tests made for
Pandora3D.
\end{enumerate}

\end{document}
